% Options for packages loaded elsewhere
\PassOptionsToPackage{unicode}{hyperref}
\PassOptionsToPackage{hyphens}{url}
\PassOptionsToPackage{dvipsnames,svgnames*,x11names*}{xcolor}
%
\documentclass[
  12pt,
]{krantz}
\usepackage{lmodern}
\usepackage{amsmath}
\usepackage{ifxetex,ifluatex}
\ifnum 0\ifxetex 1\fi\ifluatex 1\fi=0 % if pdftex
  \usepackage[T1]{fontenc}
  \usepackage[utf8]{inputenc}
  \usepackage{textcomp} % provide euro and other symbols
  \usepackage{amssymb}
\else % if luatex or xetex
  \usepackage{unicode-math}
  \defaultfontfeatures{Scale=MatchLowercase}
  \defaultfontfeatures[\rmfamily]{Ligatures=TeX,Scale=1}
\fi
% Use upquote if available, for straight quotes in verbatim environments
\IfFileExists{upquote.sty}{\usepackage{upquote}}{}
\IfFileExists{microtype.sty}{% use microtype if available
  \usepackage[]{microtype}
  \UseMicrotypeSet[protrusion]{basicmath} % disable protrusion for tt fonts
}{}
\makeatletter
\@ifundefined{KOMAClassName}{% if non-KOMA class
  \IfFileExists{parskip.sty}{%
    \usepackage{parskip}
  }{% else
    \setlength{\parindent}{0pt}
    \setlength{\parskip}{6pt plus 2pt minus 1pt}}
}{% if KOMA class
  \KOMAoptions{parskip=half}}
\makeatother
\usepackage{xcolor}
\IfFileExists{xurl.sty}{\usepackage{xurl}}{} % add URL line breaks if available
\IfFileExists{bookmark.sty}{\usepackage{bookmark}}{\usepackage{hyperref}}
\hypersetup{
  pdftitle={Die Bibel kennen: Kolosserbrief},
  colorlinks=true,
  linkcolor=Maroon,
  filecolor=Maroon,
  citecolor=Blue,
  urlcolor=Blue,
  pdfcreator={LaTeX via pandoc}}
\urlstyle{same} % disable monospaced font for URLs
\usepackage{longtable,booktabs}
\usepackage{calc} % for calculating minipage widths
% Correct order of tables after \paragraph or \subparagraph
\usepackage{etoolbox}
\makeatletter
\patchcmd\longtable{\par}{\if@noskipsec\mbox{}\fi\par}{}{}
\makeatother
% Allow footnotes in longtable head/foot
\IfFileExists{footnotehyper.sty}{\usepackage{footnotehyper}}{\usepackage{footnote}}
\makesavenoteenv{longtable}
\usepackage{graphicx}
\makeatletter
\def\maxwidth{\ifdim\Gin@nat@width>\linewidth\linewidth\else\Gin@nat@width\fi}
\def\maxheight{\ifdim\Gin@nat@height>\textheight\textheight\else\Gin@nat@height\fi}
\makeatother
% Scale images if necessary, so that they will not overflow the page
% margins by default, and it is still possible to overwrite the defaults
% using explicit options in \includegraphics[width, height, ...]{}
\setkeys{Gin}{width=\maxwidth,height=\maxheight,keepaspectratio}
% Set default figure placement to htbp
\makeatletter
\def\fps@figure{htbp}
\makeatother
\setlength{\emergencystretch}{3em} % prevent overfull lines
\providecommand{\tightlist}{%
  \setlength{\itemsep}{0pt}\setlength{\parskip}{0pt}}
\setcounter{secnumdepth}{5}
\usepackage[utf8]{inputenc}
%\usepackage[T1]{fontenc}

\usepackage[ngerman]{babel}


\usepackage{booktabs}
\usepackage{longtable}
\usepackage{layouts}
\usepackage[bf,singlelinecheck=off]{caption}


\usepackage{framed,color}
\definecolor{shadecolor}{RGB}{248,248,248}

\usepackage{float}
\floatplacement{figure}{H}
\usepackage{makeidx}
\makeindex

% The following commands make floating environments less likely 
% to float by allowing them to occupy larger fractions of pages 
% without floating.
\renewcommand{\textfraction}{0.05}
\renewcommand{\topfraction}{0.8}
\renewcommand{\bottomfraction}{0.8}
\renewcommand{\floatpagefraction}{0.75}

%Since krantz.cls provided an environment VF for quotes, we redefine the standard quote environment to VF. You can see its style in Section 2.1.

\renewenvironment{quote}{\begin{VF}}{\end{VF}}



\makeatletter
\newenvironment{kframe}{%
\medskip{}
\setlength{\fboxsep}{.8em}
 \def\at@end@of@kframe{}%
 \ifinner\ifhmode%
  \def\at@end@of@kframe{\end{minipage}}%
  \begin{minipage}{\columnwidth}%
 \fi\fi%
 \def\FrameCommand##1{\hskip\@totalleftmargin \hskip-\fboxsep
 \colorbox{shadecolor}{##1}\hskip-\fboxsep
     % There is no \\@totalrightmargin, so:
     \hskip-\linewidth \hskip-\@totalleftmargin \hskip\columnwidth}%
 \MakeFramed {\advance\hsize-\width
   \@totalleftmargin\z@ \linewidth\hsize
   \@setminipage}}%
 {\par\unskip\endMakeFramed%
 \at@end@of@kframe}
\makeatother

\makeatletter
\@ifundefined{Shaded}{
}{\renewenvironment{Shaded}{\begin{kframe}}{\end{kframe}}}
\makeatother

\newenvironment{rmdblock}[1]
  {
  \begin{itemize}
  \renewcommand{\labelitemi}{
    \raisebox{-.7\height}[0pt][0pt]{
      {\setkeys{Gin}{width=3em,keepaspectratio}\includegraphics{img/#1}}
    }
  }
  \setlength{\fboxsep}{1em}
  \begin{kframe}
  \item
  }
  {
  \end{kframe}
  \end{itemize}
  }
\newenvironment{rmdnote}
  {\begin{rmdblock}{note}}
  {\end{rmdblock}}
\newenvironment{rmdcaution}
  {\begin{rmdblock}{caution}}
  {\end{rmdblock}}
\newenvironment{rmdimportant}
  {\begin{rmdblock}{important}}
  {\end{rmdblock}}
\newenvironment{rmdtip}
  {\begin{rmdblock}{tip}}
  {\end{rmdblock}}
\newenvironment{rmdwarning}
  {\begin{rmdblock}{warning}}
  {\end{rmdblock}}
\newenvironment{rmdquestion}
  {\begin{rmdblock}{question}}
  {\end{rmdblock}}
\newenvironment{rmdyou}
  {\begin{rmdblock}{you}}
  {\end{rmdblock}}
\newenvironment{rmdobjective}
  {\begin{rmdblock}{objective}}
  {\end{rmdblock}}
\newenvironment{rmdinfo}
  {\begin{rmdblock}{caution}}
  {\end{rmdblock}}
\newenvironment{rmdbible}
  {\begin{rmdblock}{bible}}
  {\end{rmdblock}}
\newenvironment{rmdquote}
  {\begin{rmdblock}{quote}}
  {\end{rmdblock}}
\newenvironment{rmddefinition}
  {\begin{rmdblock}{definition}}
  {\end{rmdblock}}

%Then we redefine hyperlinks to be footnotes, because when the book is printed on paper, readers are not able to click on links in text. Footnotes will tell them what the actual links are.

\let\oldhref\href
\renewcommand{\href}[2]{#2\footnote{\url{#1}}}




\frontmatter
\ifluatex
  \usepackage{selnolig}  % disable illegal ligatures
\fi
\usepackage[]{natbib}
\bibliographystyle{apalike}

\title{Die Bibel kennen: Kolosserbrief}
\author{}
\date{\vspace{-2.5em}2021-03-21}

\begin{document}
\maketitle

\thispagestyle{empty}
\mainmatter

{
\hypersetup{linkcolor=}
\setcounter{tocdepth}{2}
\tableofcontents
}
\hypertarget{eine-studienreihe-uxfcber-den-kolosserbrief}{%
\chapter{Eine Studienreihe über den kolosserbrief}\label{eine-studienreihe-uxfcber-den-kolosserbrief}}

Placeholder

\hypertarget{teil_01_1_14}{%
\chapter{Kapitel 1 - Verse 1-14}\label{teil_01_1_14}}

\begin{rmdquote}
1 Paulus, Apostel Jesu Christi durch den Willen Gottes, und der Bruder
Timotheus 2 an die heiligen und treuen Brüder in Christus in Kolossä:
Gnade sei mit euch und Friede von Gott, unserem Vater, und dem Herrn
Jesus Christus! Das Gebet des Apostels für die Gemeinde 3 Wir danken dem
Gott und Vater unseres Herrn Jesus Christus, indem wir allezeit für euch
beten, 4 da wir gehört haben von eurem Glauben an Christus Jesus und von
eurer Liebe zu allen Heiligen, 5 um der Hoffnung willen, die euch
aufbewahrt ist im Himmel, von der ihr zuvor gehört habt durch das Wort
der Wahrheit des Evangeliums, 6 das zu euch gekommen ist, wie es auch in
der ganzen Welt {[}ist{]} und Frucht bringt, so wie auch in euch, von
dem Tag an, da ihr von der Gnade Gottes gehört und sie in Wahrheit
erkannt habt. 7 So habt ihr es ja auch gelernt von Epaphras, unserem
geliebten Mitknecht, der ein treuer Diener des Christus für euch ist, 8
der uns auch von eurer Liebe im Geist berichtet hat. 9 Deshalb hören wir
auch seit dem Tag, da wir es vernommen haben, nicht auf, für euch zu
beten und zu bitten, dass ihr erfüllt werdet mit der Erkenntnis seines
Willens in aller geistlichen Weisheit und Einsicht, 10 damit ihr des
Herrn würdig wandelt und ihm in allem wohlgefällig seid: in jedem guten
Werk fruchtbar und in der Erkenntnis Gottes wachsend, 11 mit aller Kraft
gestärkt gemäß der Macht seiner Herrlichkeit zu allem standhaften
Ausharren und aller Langmut, mit Freuden, 12 indem ihr dem Vater Dank
sagt, der uns tüchtig gemacht hat, teilzuhaben am Erbe der Heiligen im
Licht. 13 Er hat uns errettet aus der Herrschaft der Finsternis und hat
uns versetzt in das Reich des Sohnes seiner Liebe, 14 in dem wir die
Erlösung haben durch sein Blut, die Vergebung der Sünden.

-- Kolosser 1.1-14
\end{rmdquote}

\hypertarget{gruss-des-paulus-an-die-kolosser-kolosser-1.1-2}{%
\section{Gruss des Paulus an die Kolosser (Kolosser 1.1-2)}\label{gruss-des-paulus-an-die-kolosser-kolosser-1.1-2}}

\hypertarget{paulus-apostel-jesu-christi}{%
\subsection{\texorpdfstring{``Paulus, Apostel Jesu Christi\ldots{}''}{``Paulus, Apostel Jesu Christi\ldots''}}\label{paulus-apostel-jesu-christi}}

Damals gab es kein neues Testament. Die damaligen Christen waren auf mündliche
Überlieferungen angewiesen. Indem Paulus sich als Apostel Jesus Christi
vorstellt, zeigt er eindeutig, dass er die Autorität hat, das Wort Gottes zu
lehren, dass er im direkten Auftrag von Jesus diesen Brief schreibt.
Gott hat nämlich Paulus ``vom Mutterleib an ausgesondert'' (Galater 1.15)
und Paulus wurde von Jesus selber berufen Apostel zu werden. Das Evangelium,
was Paulus verkündigt, hat Paulus direkt vom Jesus Christus erhalten (Galater 1.11-12)

Dieser Brief Paulus
hat deshabl heute die gleiche Autorität wie damals.

\hypertarget{an-die-heiligen-und-treuen-bruxfcder-in-christus-in-kolossuxe4}{%
\subsection{\texorpdfstring{``\ldots an die heiligen und treuen Brüder in Christus in Kolossä\ldots{}''}{``\ldots an die heiligen und treuen Brüder in Christus in Kolossä\ldots''}}\label{an-die-heiligen-und-treuen-bruxfcder-in-christus-in-kolossuxe4}}

Wieso nennt Paulus die Christen in Kolossä ``Heilige'' und Brüder?

Paulus nennt sie Heilige, weil Jesus an ihre Stelle für ihre Sünden mit seinem
Blut bestraft wurde (die ``Erlösung durch {[}das{]} Blut'' Jesus, v. 14):

\begin{rmdquote}
Auch euch, die ihr einst entfremdet und feindlich gesinnt wart in den
bösen Werken, hat er jetzt versöhnt in dem Leib seines Fleisches durch
den Tod, um euch heilig und tadellos und unverklagbar darzustellen vor
seinem Angesicht

-- Kolosser 1.21-22
\end{rmdquote}

D.h. wir haben eine neue Identität in Christus (und wir sollten danach leben). Bist du keine Heiliger, wirst du nie Gott sehen (Hebräer 12.14)

Paulus nennt die Kolosser Brüder (Geschwister), weil Jesus uns, Christen, durch sein Opfer mit Gott dem Vater versöhnt hat, so dass wir Kinder Gottes sind. Wir sind eine Familie! Paulus kannte die Kolosser nicht und nahm sie als Brüder wahr und setzte sich voll ein, indem er für sie betete und sie durch seinen Brief lehrte.

\begin{rmdquestion}
Ist dir bewusst, dass du ein Heiliger bist? Dass wir Geschwister sind?
Wie siehst du die andere Christen? Die, die du nicht kennst? Als
Geschwister? Siehst du das Blut Jesus, das uns verbindet?
\end{rmdquestion}

\hypertarget{gnade-und-friede-von-gott-unserem-vater-und-dem-herrn-jesus-christus}{%
\subsection{``\ldots Gnade und Friede von Gott, unserem Vater, und dem Herrn Jesus Christus!''}\label{gnade-und-friede-von-gott-unserem-vater-und-dem-herrn-jesus-christus}}

Wünscht Paulus den Kolosser Gnade und Friede von Gott, seinem Vater und seinem Freund Jesus Christus?

Wir sollen eine persönliche Beziehung zu Gott haben, eine Vater-Kind Beziehung (Galater 4.4-7). Leider sind wir und unsere Gesellschaft vom Individualismus zu stark geprägt. Individualismus ist kein christlicher Wert (Johannes 17.20-23). Gott ist aber Gott ist kein individueller Gott. Der Gott von Abraham, Isaak, Jakob, \ldots{} Er ist nicht nur dein Vater, sondern unserem Vater. Du bist ein Teil des Körpers Christi. Unsere Beziehung zu Gott ist eng mit unserem Beziehung zu unseren Mitchristen gekoppelt. Wer Gott liebt, liebt die Anderen. Es gibt kein Solo-Christen (1. Johannes 4.7-21)

Paulus nennt Jesus ``Herr''. Da sollst du kurz über die Bedeutung vom Wort ``Herr'' nachdenken. Es geht um Gehorsamkeit gegenüber dem Boss oder dem Chef. Hier stellt sich Paulus als Sklave, Diener, Knecht Jesus vor. D.h. er verzichtet freiwillig auf seine Freiheit. Diese Art Sklaverei ist eigentlich das Beste was man haben kann, denn das Joch Jesus ist sanft (Matthäus 11.29-30).

Paulus nennt Jesus Christus der Messias. Der ersehnte Retter, die Erfüllung des Versprechens Gottes an Adam und Eva, an Abraham, etc., der rote Faden, welcher sich durch die Bibel zieht.

\begin{rmdquestion}
Wie siehst du Gott? Jesus? Wie nimmst du sie wahr? Wie haben sich unser
Gott und unser Herr offenbart?
\end{rmdquestion}

\hypertarget{paul-dankt-gott-fuxfcr-die-kolosser-fuxfcr-ihren-glauben-und-liebe-kolosser-1.3-8}{%
\section{Paul dankt Gott für die Kolosser für ihren Glauben und Liebe (Kolosser 1.3-8)}\label{paul-dankt-gott-fuxfcr-die-kolosser-fuxfcr-ihren-glauben-und-liebe-kolosser-1.3-8}}

\hypertarget{wir-danken-dem-gott-und-vater-unseres-herrn-jesus-christus-indem-wir-allezeit-fuxfcr-euch-beten}{%
\subsection{``Wir danken dem Gott und Vater unseres Herrn Jesus Christus indem wir allezeit für euch beten''}\label{wir-danken-dem-gott-und-vater-unseres-herrn-jesus-christus-indem-wir-allezeit-fuxfcr-euch-beten}}

Wem Paulus dankt? Gott unserem Vater

\begin{rmddefinition}
\textbf{Das biblische Gebetsmodell}

Zum unserem Vater beten in Name Jesus durch den Geist So hat sich Gott
offenbart.
\end{rmddefinition}

Wiederum sagt Paulus ``unseres'' Herrn und nicht ``meines Herrn''.

Gott danken, dass ist Gott loben und ehren:
- man zeigt unsere Abhängigkeit zu Gott, man bezeugt, dass die Hand Gottes am wirken ist.
- sich an das Wirken Gottes freuen, das gefällt Gott.
- der Schlüssel zu Zufriedenheit, eine Perspektivänderung

Danken ist deswegen ein zentrales Element des Gebetes:

\begin{rmdquote}
in allem lasst durch Gebet und Flehen mit Danksagung eure Anliegen vor
Gott kundwerden

-- Philipper 4.6
\end{rmdquote}

Paulus dankt erst (siehe auch 1. Korinther 1.4) bevor er Gott um etwas bittet ein Paar Verse später. Ausser in Galater und Titus dankt Paulus in all seinen Briefe Gott.

\begin{rmdquestion}
Betest du zu Gott dem Vater? Wenn du betest, wie viel Platz nimmt dein
Dank an Gott?
\end{rmdquestion}

\hypertarget{da-wir-gehuxf6rt-haben-von-eurem-glauben-an-jesus-christus-und-von-eurer-liebe-zu-allen-heiligen}{%
\subsection{\texorpdfstring{``\ldots da wir gehört haben von eurem Glauben an Jesus Christus und von eurer Liebe zu allen Heiligen\ldots{}''}{``\ldots da wir gehört haben von eurem Glauben an Jesus Christus und von eurer Liebe zu allen Heiligen\ldots''}}\label{da-wir-gehuxf6rt-haben-von-eurem-glauben-an-jesus-christus-und-von-eurer-liebe-zu-allen-heiligen}}

Wieso dankt Paulus Gott anstatt den Kolosser für ihre Bemühungen zu danken bzw. zu loben oder ermutigen?

Paulus dankt Gott dafür, dass seine Gnade im Leben der Kolosser wirksam wird. Die Betonung liegt eindeutig auf die Wirkung Gottes. Gott ist souverän. Mann vergisst schnell, dass es Gott ist, der in uns das ``Wollen sowie das Vollbringen'' (Philipper 2.13) wirkt. Ohne dass wir uns aus unserer Verantwortung entziehen können. Wie genau es funktioniert, ist ein Geheimnis. Allein schaffen wir es nicht. Die Gnade Gottes ist unsere Heiligung sowie unsere praktische Heiligung im Alltag.

Wie häufig dankt Paulus Gott für die Auswirkung der Gnade Gottes bei den Kolosser?
``Allezeit''! Das zeigt, dass man nicht aufhören sollte, dafür zu danken.

\begin{rmddefinition}
Zwei Wahrheiten, die in der Bibel enthalten sind:

\begin{enumerate}
\def\labelenumi{\arabic{enumi}.}
\tightlist
\item
  Gott ist absolut souverän, aber seine Souveränität wirkt nie so, dass
  die Verantwortung des Menschen verringert wird.
\item
  Der Mensch ist verantwortlich und wird sich für sein Handeln
  verantworten müssen. Aber die menschliche Verantwortung schwächt
  niemals die Souveränität Gottes.
\end{enumerate}
\end{rmddefinition}

Was sind Glauben und Liebe zu den Heiligen?

Glauben und Liebe sind keine abstrakten Begriffe. Damit man berichten kann, dass die Kolosser Glaube und Liebe haben, müssen Ihre Glaube und Liebe sichtbar sein, ein öffentliche Wirkung haben.

Der Glaube ist ``eine feste Zuversicht auf das, was man hofft, eine Überzeugung von Tatsachen, die man nicht sieht'' (Hebräer 11.1).
Wenn man in Hebräer 11 weiter liest, entdeckt man der Glauben der Helden des AT. Was stellt man fest? Der Glaube bewirkt gute Werke, die Gott gefallen. Das ist genau was Jakobus sagt: ``Denn gleichwie der Leib ohne Geist tot ist, also ist auch der Glaube ohne die Werke tot'' (Jakobus 2.26). Hier sieht man die Beziehung zwischen Glaube und Liebe. Die Liebe ist der Prüfstein des Glaubens. Keine Liebe, keine Glaube.

Wer sind die Heiligen? Mit Heiligen sind die Christen in dem Umkreis der Kolosser gemeint. Die Liebe zu unseren Geschwistern sollte unser Markenzeichen sein: In Johannes 13.35 sagt Jesus, dass ``jedermann {[}wird{]} erkennen, dass ihr meine Jünger seid, wenn ihr Liebe untereinander habt''. Wie drück sich die Liebe aus? Zwei Beispiele

\begin{rmdquote}
Wir ermahnen euch aber, Brüder: Verwarnt die Unordentlichen, tröstet die
Kleinmütigen, nehmt euch der Schwachen an, seid langmütig gegen
jedermann! 15 Seht darauf, dass niemand Böses mit Bösem vergilt, sondern
trachtet allezeit nach dem Guten, sowohl untereinander als auch
gegenüber jedermann!

-- 1. Thessalonicher 5.14-15
\end{rmdquote}

\begin{rmdquote}
Die Liebe ist langmütig und gütig, die Liebe beneidet nicht, die Liebe
prahlt nicht, sie bläht sich nicht auf; 5 sie ist nicht unanständig, sie
sucht nicht das Ihre, sie läßt sich nicht erbittern, sie rechnet das
Böse nicht zu; 6 sie freut sich nicht an der Ungerechtigkeit, sie freut
sich aber an der Wahrheit; 7 sie erträgt alles, sie glaubt alles, sie
hofft alles, sie erduldet alles.

-- 1. Korinther 13.4-7
\end{rmdquote}

\begin{rmdquestion}
Wie oft dankst du Gott für deine Geschwister? Dass Gott in sie wirkt?
Wie oft dankst du Gott für materielle Sachen, die für die Ewigkeit
nichts bringen.

Bewirkt dein Glaube Taten? Spürt man hier in unserer Gemeinde unsere
gegenseitige Liebe? Ist es hier in der Gemeinde anders als in der Welt?
\end{rmdquestion}

\hypertarget{um-der-hoffnung-willen-die-euch-aufbewahrt-ist-im-himmel-von-der-ihr-zuvor-gehuxf6rt-habt-durch-das-wort-der-wahrheit-des-evangeliums}{%
\subsection{``\ldots{} um der Hoffnung willen, die euch aufbewahrt ist im Himmel, von der ihr zuvor gehört habt durch das Wort der Wahrheit des Evangeliums''}\label{um-der-hoffnung-willen-die-euch-aufbewahrt-ist-im-himmel-von-der-ihr-zuvor-gehuxf6rt-habt-durch-das-wort-der-wahrheit-des-evangeliums}}

Was ist diese Hoffnung wovon Paulus redet?

Vers 12 liefert die Erklärung: die Hoffnung an die Teilhabe am ``Erbe der Heiligen im Licht''. Was heisst das konkret? Kinder Gottes (Christen) sind ``Erben Gottes und Miterben des Christus'' und werden ``mit ihm verherrlicht werden'' (Römer 8.17). Wir haben einen Schatz, eine Erbe im Himmel! Paulus sagte: ``die Leiden der jetzigen Zeit {[}fallen{]} nicht ins Gewicht {[}\ldots{]} gegenüber der Herrlichkeit, die an uns geoffenbart werden soll.'' (Römer 8.18)

Was ist die Verbindung zwischen dieser Hoffnung und dem Glaubem und der Liebe?

Die Hoffnung ist der Antrieb unseres Glaubens und unserer Liebe. Wir haben eine gemeinsame Hoffnung. Wir sind als Geschwister gemeinsam unterwegs zu Gottes Haus und wir unterstützen uns auf dem Weg, dass wir alle zusammen ankommen.

Wo liegt diese Hoffnung?

Diese Hoffnung liegt im Himmel aufbewahrt. Niemand kann uns diese Hoffnung wegnehmen. Diese Hoffnung ändert unseren Blickwinkel. Wir schauen zum Himmel und nicht mehr auf diese Welt und ihre Probleme.

Woher haben die Kolosser von dieser Hoffnung erfahren?

Im Evangelium, das ihnen von Epaphras verkündigt wurde. Wir alle wissen, wie schnell wir unsere Hoffnung auf den Himmel verlieren oder vergessen\ldots{} deshalb, kann es uns nur gut tun, sich von dem Evangelium zu ernähren, weil es eine Kraft ist!

\begin{rmdquestion}
Lebst du im Alltag in dieser Hoffnung? Was bewirkt diese Hoffnung bei
dir?
\end{rmdquestion}

\hypertarget{das-wort-der-wahrheit-des-evangeliums-das-zu-euch-gekommen-ist-wie-es-auch-in-der-ganzen-welt-ist-und-frucht-bringt-so-wie-auch-in-euch-von-dem-tag-an-da-ihr-von-der-gnade-gottes-gehuxf6rt-und-sie-in-wahrheit-erkannt-habt.}{%
\subsection{``\ldots das Wort der Wahrheit des Evangeliums, das zu euch gekommen ist, wie es auch in der ganzen Welt {[}ist{]} und Frucht bringt, so wie auch in euch, von dem Tag an, da ihr von der Gnade Gottes gehört und sie in Wahrheit erkannt habt.''}\label{das-wort-der-wahrheit-des-evangeliums-das-zu-euch-gekommen-ist-wie-es-auch-in-der-ganzen-welt-ist-und-frucht-bringt-so-wie-auch-in-euch-von-dem-tag-an-da-ihr-von-der-gnade-gottes-gehuxf6rt-und-sie-in-wahrheit-erkannt-habt.}}

Was ist das Evangelium?

Das \textbf{Evangelium} ist die gute Nachricht. Damals, als es noch kein neues Testament gab, waren die Apostel und ihre Mitarbeiter die Träger des Evangeliums.
In unserer postmoderne Welt ist die Wahrheit relativ. Das Evangelium ist aber ein \textbf{Wort der Wahrheit}, woran man sich orientieren kann.
Das Evangelium ist eine Kraft, die in den Menschen wirkt und Früchte trägt. Diese Kraft ist weder magisch noch von uns beherrschbar. Das ist der Samen, der auf die gute Erde fällt und Früchte trägt (Markus 4.8). Die Früchte, die Paulus erwähnt, sind u.a. der Glaube, die Liebe und die Hoffnung. Gott wirkt durch sein Wort.

Was meint Paulus mit dem Ausdruck ``das Evangelium, das in der ganzen Welt ist und Früchte bringt''?

Das kann eine Anspielung an den ursprünglichen Plan Gottes sein. Bei der Schöpfung befahl Gott den Menschen fruchtbar zu sein, sich zu mehren und die Erde zu füllen (1. Mose 1.26-28). Die Erde soll gefüllt werden mit Menschen nach dem Bild Gottes. Als Adam und Eva gegen Gott rebellierten, drangen Sünde und Tod ein und verwüsteten die Welt Gottes. Durch das Evangelium erschafft Gott jedoch ein Volk von Ebenbildern:

\begin{rmdquote}
Denn die er zuvor ersehen hat, die hat er auch vorherbestimmt, dem
Ebenbild seines Sohnes gleichgestaltet zu werden, damit er der
Erstgeborene sei unter vielen Brüdern.

-- Römer 8.29
\end{rmdquote}

Das ist der Rettungsplan Gottes, er stellt seinen Plan wieder her. Er verwandelt uns wieder in das, was wir hätten sein sollen.

\hypertarget{so-habt-ihr-es-ja-auch-gelernt-von-epaphras-unserem-geliebten-mitknecht-der-ein-treuer-diener-des-christus-fuxfcr-euch-ist-8der-uns-auch-von-eurer-liebe-im-geist-berichtet-hat.}{%
\subsection{``\ldots So habt ihr es ja auch gelernt von Epaphras, unserem geliebten Mitknecht, der ein treuer Diener des Christus für euch ist, 8der uns auch von eurer Liebe im Geist berichtet hat.''}\label{so-habt-ihr-es-ja-auch-gelernt-von-epaphras-unserem-geliebten-mitknecht-der-ein-treuer-diener-des-christus-fuxfcr-euch-ist-8der-uns-auch-von-eurer-liebe-im-geist-berichtet-hat.}}

W as ist ein treuer Diener des Christus?

\begin{rmdquote}
Epaphras, der einer der Euren ist, ein Knecht des Christus, der allezeit
in den Gebeten für euch kämpft, damit ihr fest steht, vollkommen und zur
Fülle gebracht in allem, was der Wille Gottes ist.

-- Kolosser 4.12
\end{rmdquote}

\begin{rmdquestion}
Bist du ein treuer Diener von Jesus? Kämpfst du für uns, indem du
betest, dass wir Gott ähnlicher werden?
\end{rmdquestion}

  \bibliography{book.bib,packages.bib}

\printindex

\end{document}
